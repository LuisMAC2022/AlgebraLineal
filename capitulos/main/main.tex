\documentclass{IEEEtran}


\usepackage{enumitem}
\usepackage{hyperref}
\usepackage{amssymb}
\usepackage{amsmath}

\title{Algebra Lineal}
\author{Profesor: Jes\'{u}s Gil Galindo Cuevas\\Notas: LaRana}
\date{Enero - Febrero de 2025}
\newcommand{\is}[2]{\ensuremath{{#1}\{{#2}\}}}
\newcommand{\isp}[2]{\ensuremath{{#1}({#2})}}
\newcommand{\be}[2]{\begin{#1}{#2}\end{#1}}
\newcommand{\mb}[1]{\mathbb{#1}}
\newcommand{\ul}[1]{\underline{#1}}
\newcommand{\ol}[1]{\overline{#1}}
\newcommand{\sumin}{\sum_{i=1}^n}
\newtheorem{teo}{Teorema}
\newtheorem{defin}{Definici\'{o}n}
\newtheorem{poof}{Demostraci\'{o}n}
\newtheorem{ejem}{Ejemplo}
\newtheorem{propo}{Proposici\'{o}n}
\newtheorem{nota}{Nota}
\newtheorem{conv}{Conversaci\'{o}n}


\begin{document}
\maketitle

\documentclass{IEEEtran}


\usepackage[spanish]{babel}
\usepackage{enumitem}
\usepackage{hyperref}



\title{Algebra Lineal}
\author{Profesor: Jes\'{u}s Gil Galindo Cuevas\\Notas: Luis Daniel Arana}
\date{Febrero 2024}

%\newcommand{}{}
%\newtheorem

\begin{document}
\maketitle
\section*{Temario}
\begin{enumerate}
	\item Espacios Vectoriales
	\begin{enumerate}[label*=\arabic*]
	\item Campos	
	\item Espacios
	\item Subespacios
	\item Independencia Lineal
	\item Base
	\item Dimensi\'{o}n
\end{enumerate}
	\item Transformaciones Lineales
\begin{enumerate}[label*=\arabic*]
	\item Definici\'{o}n y propiedades
	\item Isomorfismo
	\item N\'{u}cleo e imagen
	\item Representaci\'{o}n Matricial
\end{enumerate}
	\item Eigenvalor y eigenvector
\begin{enumerate}[label*=\arabic*]
	\item Definiciones y polinomio caracteristico 
	\item Teorema de Cayley Hamilton
	\item Diagonalizaci\'{o}n de operaciones lineales
\end{enumerate}
	\item Producto Interno
\begin{enumerate}[label*=\arabic*]
	\item Espacios con producto interno 
	\item Ortogonalizaci\'{o}n  
	\item Proceso GramSchmidt
\end{enumerate}
\item Operadores Ortogonales
\begin{enumerate}[label*=\arabic*]
	\item Diagonalizaci\'{o}n
	\item Operadores sim\'{e}tricos
	\item Operadores hermitianos
\end{enumerate}
\end{enumerate}
%\input{carpeta/capitulo}
\section*{Literatura}
\begin{itemize}
	\item Larson
	\item Anton
	\item Grossman
	\item Lipschutz
	\item Friedberg
	\item Hoffman
	\item Sheldon Axler
	\item Egor Maximenko, Algebra 2 y 3
\end{itemize}
\section*{Recursos adicionales}
\begin{itemize}
\item\href{jgwalneko.github.io}{Pagina web del profesor }
\item\href{youtube.com/@jg_walneko}{Canal de Youtube }
\end{itemize}
\end{document}



\newpage

%%\documentclass{IEEEtran}


%\usepackage{amssymb}

%\usepackage{amsmath}

%\title{Introducci\'{o}n}
%\author{}
%\date{27 de Enero de 2025}

%\newcommand{\mb}[1]{\mathbb{#1}}
%\newcommand{\ul}[1]{\underline{#1}}
%\newcommand{\ol}[1]{\overline{#1}}
%\newcommand{\sumin}{\sum_{i=1}^n}
%\newtheorem{teo}{Teorema}
%\newtheorem{poof}{Demostraci\'{o}n}
%\newtheorem{teo}{Teorema}

%b - brackets matrix/
% \[  \begin{bmatrix} 0 & 0 & 0\\ 0 & 0 & 0 \end{bmatrix} \]
%%v - vertical line matrix/
% \[  \begin{vmatrix} 0 & 0 & 0\\ 0 & 0 & 0 \end{vmatrix} \]
 %V - double vertical line  matrix/
%
% \[  \begin{Vmatrix} 0 & 0 & 0\\ 0 & 0 & 0 \end{Vmatrix} \]
%p - parenthesis matrix/
% \[  \begin{pmatrix} 0 & 0 & 0\\ 0 & 0 & 0 \end{pmatrix} \]

%\[ \left\{ \begin{array}{ccc}  \end{array}        \right\} \]
%

%\begin{document}
%\maketitle
\onecolumn
\section*{Introducci\'{o}n}
Iniciaremos el curso estudiando la estructura matem\'{a}tica del  espacio vectorial.\\ Para esto utilizaremos lenguaje de teor\'{i}a de conjuntos.\\Un espacio vectorial puede ser definido como una estructura:
\begin{align*}
(V, F, +, \cdot)\\
V \text{ es un vector. }\\
F \text{ es un campo.}\\
+ \text{ es un operador.}\\
\cdot \text{ es una funci\'{o}n.}\\
+ \land \cdot \text{ son binarios.}
\end{align*}
Desarrollaremos las siguientes propiedades:

\begin{enumerate}
	\item \begin{align*}u+v \in F\end{align*}
	\item \begin{align*}u+v = v+u\end{align*}
	\item \begin{align*}v + (v + w) = (u + v) + w\end{align*}
	\item \begin{align*}\exists! 0 \in F \to u + 0 = 0 + u = u\end{align*}
	\item \begin{align*}\forall u \in F \exists! -u \in F \to  u + (-u) = -u + u = 0 \end{align*}
	\item \begin{align*}uv \in F\end{align*}
	\item \begin{align*}uv = vu\end{align*}
	\item \begin{align*}u(vw) = (uv)w\end{align*}
	\item \begin{align*}\exists! 1\in F \to u\cdot 1 = \cdot 1 = u\end{align*}
	\item \begin{align*}\forall u \in F; u \neq 0 \exists! u^{-1}\in F \to uu^{-1} = u^{-1}u = 1\end{align*}
	\item \begin{align*}u(v+w)=uv+uw \land (v+w)u = vu + wu \end{align*}
	
\end{enumerate}
%input{carpeta/capitulo}
%\end{document}


%\documentclass{IEEEtran}


%\usepackage{amssymb}

%\usepackage{amsmath}

%\title{Introducci\'{o}n}
%\author{}
%\date{27 de Enero de 2025}
%
%\newcommand{\mb}[1]{\mathbb{#1}}
%\newcommand{\ul}[1]{\underline{#1}}
%\newcommand{\ol}[1]{\overline{#1}}
%\newcommand{\sumin}{\sum_{i=1}^n}
%\newtheorem{teo}{Teorema}
%\newtheorem{poof}{Demostraci\'{o}n}
%%\newtheorem{teo}{Teorema}
%
%%b - brackets matrix/
%% \[  \begin{bmatrix} 0 & 0 & 0\\ 0 & 0 & 0 \end{bmatrix} \]
%%%v - vertical line matrix/
%% \[  \begin{vmatrix} 0 & 0 & 0\\ 0 & 0 & 0 \end{vmatrix} \]
% %V - double vertical line  matrix/
%%
%% \[  \begin{Vmatrix} 0 & 0 & 0\\ 0 & 0 & 0 \end{Vmatrix} \]
%%p - parenthesis matrix/
%% \[  \begin{pmatrix} 0 & 0 & 0\\ 0 & 0 & 0 \end{pmatrix} \]
%
%%\[ \left\{ \begin{array}{ccc}  \end{array}        \right\} \]
%%
%
%\begin{document}
%\maketitle
\section*{Introducci\'{o}n}
Iniciaremos el curso estudiando la estructura matem\'{a}tica del  espacio vectorial. Para esto utilizaremos lenguaje de teor\'{i}a de conjuntos.\\Un espacio vectorial puede ser definido como una estructura:
\begin{align*}
(V, F, +, \cdot)\\
V \text{ es un vector. }\\
F \text{ es un campo.}\\
+ \text{ es un operador.}\\
\cdot \text{ es una funci\'{o}n.}\\
+ \land \cdot \text{ son binarios.}
\end{align*}
Desarrollaremos las siguientes propiedades:

\begin{enumerate}
	\item \begin{align*}u+v \in F\end{align*}
	\item \begin{align*}u+v = v+u\end{align*}
	\item \begin{align*}v + (v + w) = (u + v) + w\end{align*}
	\item \begin{align*}\exists! 0 \in F \to u + 0 = 0 + u = u\end{align*}
	\item \begin{align*}\forall u \in F \exists! -u \in F \to  u + (-u) = -u + u = 0 \end{align*}
	\item \begin{align*}uv \in F\end{align*}
	\item \begin{align*}uv = vu\end{align*}
	\item \begin{align*}u(vw) = (uv)w\end{align*}
	\item \begin{align*}\exists! 1\in F \to u\cdot 1 = \cdot 1 = u\end{align*}
	\item \begin{align*}\forall u \in F; u \neq 0 \exists! u^{-1}\in F \to uu^{-1} = u^{-1}u = 1\end{align*}
	\item \begin{align*}u(v+w)=uv+uw \land (v+w)u = vu + wu \end{align*}
	
\end{enumerate}
%input{carpeta/capitulo}
%\end{document}



\newpage
\twocolumn
%\documentclass{IEEEtran}
%
%
%\usepackage{amssymb}
%
%\usepackage{amsmath}
%
%\title{}
%\author{Notas: Luis Daniel Arana}
%\date{Febrero de 2025}
%
%\newcommand{\mb}[1]{\mathbb{#1}}
%\newcommand{\ul}[1]{\underline{#1}}
%\newcommand{\ol}[1]{\overline{#1}}
%\newcommand{\sumin}{\sum_{i=1}^n}
%\newtheorem{teo}{Teorema}
%\newtheorem{poof}{Demostraci\'{o}n}
%%\

%b 
% \
%%v
% \
 %V
%
% \
%p 
% \


%\begin{document}
\section{Axiomas del Campo}
\begin{description}
	\item[Cerradura de la suma del campo]$\forall u,v \in V;\\ u+v \in V$
	\item[Asociatividad de la suma del campo]$\forall u,v,w \in V;\\ u+(v+w) = (u+v)+w $ 
	\item[Conmutatividad de la suma del campo]$\forall u,v \in V;\\ u+v = v+u $
	\item[Elemento neutro en el campo]$\exists! 0 \in V$ tal que: \\ $  0 + v = v+0 = v $
	\item[Inverso aditivo en el campo]$\forall v \in V \  \exists! w \in V$ tal que: \\$ v + w = w + v = 0 $ 
	\item [Cerradura del producto en el campo] $\forall \lambda \in \mb{K};\forall v \in V$ ; \\ $\lambda v \in V$
	\item[Distributividad respecto escalar] $\forall \alpha , \beta \in \mb{K};$ \\ $\forall v \in V; (\alpha + \beta )v = \alpha v + \beta v)$
	\item [Distributividad respecto elemento vector] $\forall \alpha \in \mb{K};$ \\ $  \forall u,v \in V; \alpha (u+v)= \alpha u + \alpha v$  
	\item [Asociatividad] $\forall \alpha, \beta \in \mb{K};$ \\ $\forall v \in V; (\alpha\beta)v = \alpha(\beta v) $
	\item [Identidad de $\mb{K}$ respecto al producto]$\forall u \in V; \exists! 1 \in \mb{K}$\\ tal que:  $1u= u$
\end{description}	
\begin{defin}
Sean $A,B$ conjuntos.\\ Decimos que:\\
$A \subseteq B$ \\
Si: \\
$\forall x \in A \rightarrow x \in B$
\end{defin}

%\input{carpeta/capitulo}
%\end{document}



%\onecolumn
%\documentclass{IEEEtran}
%
%
%\usepackage{amssymb}
%
%\usepackage{amsmath}
%
%\title{}
%\author{Notas: Luis Daniel Arana}
%\date{Febrero de 2025}
%
%\newcommand{\mb}[1]{\mathbb{#1}}
%\newcommand{\ul}[1]{\underline{#1}}
%\newcommand{\ol}[1]{\overline{#1}}
%\newcommand{\sumin}{\sum_{i=1}^n}
%\newtheorem{teo}{Teorema}
%\newtheorem{poof}{Demostraci\'{o}n}
%%\newtheorem{teo}{Teorema}
%
%%b - brackets matrix/
%% \[  \begin{bmatrix} 0 & 0 & 0\\ 0 & 0 & 0 \end{bmatrix} \]
%%%v - vertical line matrix/
%% \[  \begin{vmatrix} 0 & 0 & 0\\ 0 & 0 & 0 \end{vmatrix} \]
% %V - double vertical line  matrix/
%%
%% \[  \begin{Vmatrix} 0 & 0 & 0\\ 0 & 0 & 0 \end{Vmatrix} \]
%%p - parenthesis matrix/
%% \[  \begin{pmatrix} 0 & 0 & 0\\ 0 & 0 & 0 \end{pmatrix} \]
%
%%\[ \left\{ \begin{array}{ccc}  \end{array}        \right\} \]

%\begin{document}
%

\section{Espacios Vectoriales}
\begin{defin}
	(Espacio Vectorial)\\
Sea: \\
$V$ un conjunto no vac\'{i}o,\\ $\mb{F}$ un campo y \\  $+: V\times V \to V;$ \\$ \cdot : \mb{F}\times V \to V$ \\
Tal que se cumplen los siguientes axiomas:

\begin{description}
	\item[Cerradura de la suma]
\item[Asociatividad de la suma]
\item[Conmutatividad de la suma]
\item[Existencia del vector cero]
\item[Existencia del inverso aditivo]
\item[Cerradura del producto escalar sobre el vector]
\item[Distributividad del producto escalar respecto a la suma del vector]
\item[Distributividad del producto escalar respecto a la suma del campo ]
\item[Existencia del neutro multiplicativo del campo]
\item[Asociatividad del producto de escalares]
\end{description}

Entonces $V$ se llama espacio vectorial sobre el campo $\mb{F}$. A los elementos de $V$ se le llama vectores.

\begin{ejem}
$$
\end{enumerate}


\end{ejem}

\end{defin}
%%\input{carpeta/capitulo}
%\end{document}
%

%\documentclass{IEEEtran}
%
%
%\usepackage{amssymb}
%
%\usepackage{amsmath}
%
%\title{}
%\author{Notas: Luis Daniel Arana}
%\date{Febrero de 2025}
%
%\newcommand{\mb}[1]{\mathbb{#1}}
%\newcommand{\ul}[1]{\underline{#1}}
%\newcommand{\ol}[1]{\overline{#1}}
%\newcommand{\sumin}{\sum_{i=1}^n}
%\newtheorem{teo}{Teorema}
%\newtheorem{poof}{Demostraci\'{o}n}
%%\newtheorem{teo}{Teorema}
%
%%b - brackets matrix/
%% \[  \begin{bmatrix} 0 & 0 & 0\\ 0 & 0 & 0 \end{bmatrix} \]
%%%v - vertical line matrix/
%% \[  \begin{vmatrix} 0 & 0 & 0\\ 0 & 0 & 0 \end{vmatrix} \]
% %V - double vertical line  matrix/
%%
%% \[  \begin{Vmatrix} 0 & 0 & 0\\ 0 & 0 & 0 \end{Vmatrix} \]
%%p - parenthesis matrix/
%% \[  \begin{pmatrix} 0 & 0 & 0\\ 0 & 0 & 0 \end{pmatrix} \]
%
%%\[ \left\{ \begin{array}{ccc}  \end{array}        \right\} \]
%%
%%
%%
%%
%%
%%
%%
%%
%
\section{Subespacios Vectoriales}
\be{defin}{(Subespacio Vectorial)\\ Sea $V$ un espacio vectorial sobre $\mb{K}$. Sea $W\subseteq V; W \neq \emptyset$. Decimos que $W$ es subespacio vectorial de $V$ si $W$ constituye un espacio vectorial por si mismo con el campo $\mb{K}$ y las mismas operaciones de $V$.} Ejemplos: \be{itemize}{\item Todo subespacio de $\mb{R}^2 \text{ sobre } \mb{KR} $ \item $\mb{R}^2$ \item \is{}{\isp{}{0,0}\in \mb{R}^2} \item Cualquier conjunto de puntos en $\mb{R}^2$ que forme una recta que pase por el origen \is{}{x,mx \in\mb{R}^2}}

\be{propo}{(Propiedades de los subespacios vectoriales)\\Sea $V$ un espacio vectorial sobre $\mb{K}$ y sea $C \in \mb{K}$ Entonces se cumplen: \be{enumerate}{\item $0v=0$\item $\alpha0 = 0$\item Si $\alpha v=0\Rightarrow  \alpha = 0 \lor v=0$\item$(-1)v= -v$} }
\be{poof}{}
\be{teo}{(Criterio de subespacio)\\ Sean $V$ un espacio vectorial sobre $\mb{K}$ y CRITERIO DIFF from Subesp?? }
%
%
%
%
%
%
%\begin{document}
%
%%\input{carpeta/capitulo}
%\end{document}
%

\newpage
\begin{defin} (Combinaci\'{o}n Lineal)

    Sea:\\ $ S =\{v_1,.., v_k\} %un subconjunto de un espacio vectorial%
    \subset V  \land c_1,..,c_k \in \mathbb{K}$\\ Se dice que un vector $v \in V  $ es una combinaci\'{o}n lineal de los vectores en S si v se puede escribir como  $v =\sum{_{k=1}^nc_kv_k }$
\end{defin}



\begin{ejem}
    Sea:\\ $S=\{v_1,v_2\} \subset \mathbb{R}^2$\\ \text{donde:} \\$v_1 =(4,-3) \land v_2 = (-1,9).   $

    Escriba, de ser posible a $v=(6,-18)\in \mathbb{R}^2$ como una cominaci\'{o}n lineal de $v_1$ y $v_2$.

    Soluci\'{o}n:
    Supongamos que existen $c_1,c_2 \in \mathbb{R}$ tal que $v = c_1v_1 + c_2v_2$


    $ (6,-18) = c_1(4,-3)+c_2(-1,9) $ \\$ (6,-18) = (4c_1-c_2,-3c_1+9c_2) $ \\ Por igualdad de parejas ordenadas tenemos que: \\

    \{ $(4c_1 + 9c_2 = 6)9; -3c_1 + 9c_2=-18$\} $\to 36c_1-9c_2 = 54 ; -3c_1 + 9c_2 = -18 \to 33c_1 =36 \to c_1 = \frac{12}{11}  $ Tarea sustituir $c_1$
\end{ejem}

    \begin{ejem}

     Escriba, de ser posible, a $v=(3,\frac{1}{2})\in \mathbb{R}^2$ como una combinacion lineal de $v_1=(-2,3),v_2=(4,-6) \in \mathbb{R}^2$

    Solucion:

    Supongamos que existen $k_1,k_2 \in \mathbb{R}$ tal que $v= k_1,v_1 + k_2v_2$
    \\
    $(3,\frac{1}{2}=k_1(-2,3)+k_2(4,-6))$ \\
     $(3,\frac{1}{2}=(-2k_1,3k_1)+(4k_2,-6k_2))$ \\
      $(3,\frac{1}{2}=(-2k_1 + 4k_2,3k_1 -6k_2))$ \\

      Por igualdad de parejas ordenadas tenemos:

      $\{(-2k_1 + 4k_2=3)3; (3k_1 -6k_2 = \frac{1}{2})2\} \to \{-6k_1+12k_2=9;6k_1-12k_2=2\}\to 0=10 $

      Se llega a una contradiccion
\end{ejem}

\begin{ejem}

 Encuentre el valor de $p\in \mathbb{R} \text{ para que } v=(1,p)\in \mathbb{R}^2 $ sea combinaci\'{o}n lineal de $v_1=(2,3); v_2=(-1,4)$.

Soluci\'{o}n: Sean $\alpha,\beta \in \mathbb{R} $ tal que $v=\alpha v_1 + \beta v_2$\\ $(1,p) = \alpha(2,3)+\beta (-1,4)$\\ $(1,p) = (2\alpha,3\alpha)+ (-1\beta,4\beta)$\\

Se genera el sistema:

$\{(2\alpha-\beta = 1)4;3\alpha+4\beta = p\} \to (8\alpha - 4\beta = 4) + (3\alpha+4\beta = p) = 11\alpha = p + 4 \to \alpha = \frac{p+4}{11} $

Sustituimos

$(3\frac{p+4}{11}+4\beta = p)\to 4\beta = p -\frac{3p+12}{11} \to 4\beta = \frac{11p - 3p-12}{11} \to \beta = \frac{8p -12}{44} \to \beta = \frac{2p -3}{11}; p\in \mathbb{R} $

Si $p=7 \to \alpha = 1; \beta = 1$

Si $p=0, \text{ tenemos } \alpha=\frac{4}{11}; \beta=\frac{-3}{11}$

Soluci\'{o}n: Si $p\in \mathbb{R}, (1,p) = \frac{p+4}{11} (2,3) + \frac{2p-3}{11}(-1,4)$. Tarea: Comprueba la proposici\'{o}n anterior.
\end{ejem}



\begin{ejem}

 Escriba, de ser posible

$v = [\begin{matrix}
    1 & 2 & 3 \\ 4 & 5 & 6
\end{matrix} ] \in M_{2x2}(\mathbb{R}) $

Como una combinaci\'{o}n lineal de $v_1,v_2,v_3,v_4,\in M_{2x2}(\mathbb{R})$ donde $v_1 = [\begin{matrix}
    1 & 0  \\ 0 & -1
\end{matrix} ] , v_2 = [\begin{matrix}
    0 & 1  \\ 0 & 0
\end{matrix} ], v_3 = [\begin{matrix}
    1 & 1  \\ 1 & 0
\end{matrix} ] \text{ y } v_4 = [\begin{matrix}
    -1 & 0  \\ 0 & -1
\end{matrix} ] $


Soluci\'{o}n: Supongamos que existen $a_1,a_2,a_3,a_4\in\mathbb{R}$ tal que $v =a_1v_1+a_2v_2+a_3v_3+a_4v_4$


$    [\begin{matrix}
    2 & -3  \\  0& 4
\end{matrix} ]  = a_1[\begin{matrix}
    1 & 0  \\ 0 & -1
\end{matrix} ] + 2_2 [\begin{matrix}
    0 & 1  \\ 0 & 0
\end{matrix} ]+ a_3[\begin{matrix}
    1 & 1  \\ 1 & 0
\end{matrix} ] +  a_4[\begin{matrix}
    -1 & 0  \\ 0 & -1
\end{matrix} ] $


$\{a_1 + a_3 -a_4 = 2...(1);a_2+a_3 =-3...(2); a_3 = 0...(3); -a_1 -a_4 = 4...(4)  \} $

De (3), $a=0$. Sustituir $a_3 = 0 $ en (2): $a_2=-3$

Sustituir $a_3 = 0$ en (1) y resolver junto (4):

$(a_1 -a_4=2)+ (-a_1 -a_4=4)=-2a_4=6 \to a_4=-3; a_1=-1.  $ Tarea sustituir y comprobar.

\end{ejem}

\begin{ejem}

De ser posible, escriba a $p(x)=2x^2-6x+9$  como una combinaci\'{o}n lineal de los vectores $u(x)= 1-x, v(x)=3x^2, w(x)=5+x^2\in P_2(\mathbb{R})$

Soluci\'{o}n: Sean  $a,b,c\in\mathbb{R}$ tal que $p(x)=au(x)+bv(x)+cw(x)$

$2x^2-6x+9=a(1-x)+b(3x^2)+c(5+x^2) = a-ax+3bx^2=5c=cx^2$

$(a+5c)-ax+(3b+c)x^2$

Por igualdad de polinomios tenemos $\{2=3b+c; -6=-a \to a=6; 9 = a+5c\to c=\frac{3}{5}\}\to 3b+c=2 \to b=\frac{7}{15}$

As\'{i}: $2x^2-6x+9=6(1-x)+\frac{7}{15}(3x^2)+\frac{3}{5}(5+x^2)$

$=6-6x+\frac{21}{15}x^2+\frac{3}{5}x^2$

$=2x^2-6x+9$

\end{ejem}

$==================================================================$
\begin{teo}

Envoltura o envolvente lineal o espacio generado)

Sean $v1,..,v_k$  elementos de un espacio vectorial $V$ sobre el campo $\mathbb{K}$ . El conjunto de todas las combinaciones lineales que se pueden formar con $v_j \leq j \leq k$ , es un subespacio de $V$ y se denomina: envoltura lineal, envolvente lineal o espacio generado por los $v_j$.
\end{teo}


\begin{poof}

    Denotemos por $gen=\{v_1,..,v_k\} $ al conjunto de todas las combinaciones lineales de los $v_j$.  Demostrar que no es vacio y es subconjunto de $V$.

   Idea: Si $\alpha v_1+...+\alpha_kv_k\in V$ por cerradura; Si $\beta v_1+...+\beta_kv_k\in V$ por cerradura. Entonces $(\alpha v_1+...+\alpha_kv_k) \cdot (\beta v_1+...+\beta_k v_k)$ por cerradura del producto.

La unica combinaci\'{o}n lineal que puede formar un subespacio es el cero, que formaria el subespacio trivial.

Regresando a la demostraci\'{o}n:

    Como $v_1,..,v_k \in gen\{..\} $ entonces $gen\{..\}\neq \emptyset $ y $gen\{..\} \subset V$

    Aplicando el cr\'{i}terio de subespacio para verificar la cerradura de las operaciones:

    sean $\alpha v_1+...+\alpha_kv_k;\beta v_1+...+\beta_kv_k \in gen$ y $\lambda\in\mathbb{K}.$ Entonces:

    $(\alpha_1 v_1+...+\alpha_kv_k) + (\beta_1 v_1+...+\beta_k v_k) = (\alpha_1 + \beta_1)v_1+..+(\alpha_k+\beta_k)v_k \in gen$ por asociatividad de la suma en $V$ y distributividad de la suma en $V$   y luego por la cerradura de la suma en $\mathbb{K}$.

    La otra cerradura se deja como ejercicio.

    Ejemplo: Sea $v=(1,1)\in\mathbb{R}^2$.

    Este vector por si solo no forma un espacio vectorial. Salvo si fuera el vector 0. Sin embargo, al formar todas las combinaciones lineales, si genera un subespacio. Entonces
    $gen\{v\}$ es un subespacio de $\mathbb{R}^2$.

    Geometricamente seria todos los puntos de la diagonal que pasa por el origen(al ser espacio vectorial) y por el punto $v$. $gen\{(1,1)\}=\{(x,x)\in\mathbb{R}^2\}$.

    Otra forma de escribir el generado es: $gen\{(1,1)\}=\{\lambda(1,1):\lambda\in\mathbb{R}\}$.

    Notaci\'{o}n: Sea $S\subset V; genS =linS = \mathcal{L}(S) = span(S) = \ell(S)$

\end{poof}

Ejercicio: La union de 2 subespacios es siempre un subespacio?

$V=\mathbb{R}^2; W_1=\{(x,0)\in\mathbb{R}^2\}; W_2 =\{(0,y)\in\mathbb{R}^2\}$

Recordemos que $W_1 \cup W_2= \{ \mathbb{R};\mathbb{R}^2;\{(0,0)\}\}$

Podemos definir como $W_1 \cup W_2= (x,0) \lor (0,x) : x\in\mathbb{R}\}$

$(1,0)+(0,1) = (1,1)  \in$

La intereseccion finita de un subespacio con otro subespacio es un subespacio. Una interseccion de un infiniro numerable es un subespacio. La interseccion de un infinito no numerable de subespacios tambien es un subespacio(?). Proposici\'{o}n: La interseccion arbitraria de subespacios es un subespacio.

\begin{nota}

Estudiar infinitos numerables y Analisis matematicos de Rudin, demostracion de George Cantor sobre inumerabilidad de los numeros Reales, no se puede hacer una funcion buyectiva de los naturales.
\end{nota}

%\begin{figure}
%    \centering
%    \includegraphics[width=1\linewidth]{image.png}
%    \caption{Meme}
%    \label{fig:enter-label}
%\end{figure}


\begin{propo}



Pensemos en un espacio vectorial $V$ y un subespacio $W_1,W_2 \in V$

$W_1 \cap W_2 = \{w\in W_1  \land w \in W_2\}\subset V$
\end{propo}

\begin{propo}
La intersecci\'{o}n  arbitraria de subespacios es un subespacio.

Sea $V$ un espacio vectorial sobre $\mathbb{K}$

y sea $\{W_j\}$ una colleci\'{o}n de subespacios de $V$.

Entonces:

$\cap_{j\in J} W_j$ es un subespacio de $V$. $J$ es un conjunto arbitrario.

\end{propo}

\begin{poof}

    Como el neutro aditivo\\ (o vector cero) de $V,0_v\in W_j \forall j\in J$\\
    Entonces:\\ $0_v \in \cap_{j\in J} W_j$,\\
    luego:\\ $\cap_{j\in J} W_j \neq \emptyset$.\\
    Adem\'{a}s, todos los\\
    $W_j \subset V \to \cap W_j \subset V \forall j\in J$\\
    \textbf{Apliquemos el criterio de subespacio}\\
    Sean:
    $u,v \in \cap_{j\in J} W_j; \lambda\in\mathbb{K} $\\
    Como: $u,v \in \cap_{j\in J} W_j \to u,v \in W_j \forall j\in J$\\
    Pero $\forall j\in J,W_j $\\
    es cerrado bajo suma y producto por escalar de $V$.\\
    Entonces:\\
    $u+v\in W_j \forall j\in J \to u+v \in \cap _{j\in J} W_j$\\
    Por otro lado:\\
    $\lambda u \in W_j \forall j\in J$\\
    Luego: \\$\lambda u \in \cap_{j\in J}W_j$

\end{poof}

\begin{propo}
    El conjunto soluci\'{o}n de un sistema homog\'{e}neo es un subespacio.\\

    Sea $A\in M_{m\times n}(\mathbb{K}).$
    El conjunto soluci\'{o}n del sistema:
    $AX=0$\\
    con $X=[x_1,..,x_n]^T\in M_{n\times 1}(\mathbb{K})$
    y $0=[0,..,0]^T\in M_{m\times 1}(\mathbb{K}),$\\ es un subespacio de $\mathbb{K}^2$
\end{propo}

\begin{ejem}
   \[ \left\{ \begin{array}{c}
        x-2y+3z=0 \\
         4z=0
   \end{array}  \right\}\] Entonces: \\
    $ z=0 \land x-2y = 0; x=2y$\\
    Conjunto soluci\'{o}n:
    $S=\{(2y,y,0)\in \mathbb{R}^3\}$


\end{ejem}




%\section{Base}

\be{defin}{(Base de un espacio vectorial) \\
	Sea $V$ un espacio vectorial sobre el campo $\mb{K} \text{y sea} \is{B=}{v_1,v_2,...,v_n}subset V$. 
	Se dice que $B$ es una base para $V$ si satisface:
	\be{enumerate}{
		\item$B$ genera a $V$
		\item$B$ es conjunto linealmente independiente }
		}
	\be{ejem}{$V$ es $\mb{R}^2$ En $\mb{R}^2$, sea \is{B=}{(1,0),(0,1)}\\
		Probemos que $B$ es una base para $V$.
		\\ Veamos si $B$ genera a $V$:
		\be{align*}{\text{Sea } \isp{v=}{x,y}\in\mb{R}^2. \\
			\text{Proponemos: } c_1,c_2\in\mb{R} \text{ tal que: } \\
			v=c_1v_1 + c_2v_2 \\
			(x,y)= c_1(1,0)+c_2(0,1)\\
			(x,y)= (c_1,0)+(0,c_2)\\
			(x,y)= (c_1,c_2)\\
			\text{Por igualdad de tuplas, tenemos:}\\
			}
		}
Ahora veamos la independencia lineal: \\
Proponemos $\alpha_1\alpha_2 \in \mb{R}$ tal que:
\be{align*}{
	\alpha_1v_1+\alpha_2v_2=0\\
	\isp{\alpha_1(1,0)+\alpha_2(0,1)=}{0,0}\\
	\isp{(\alpha_1,0)+(0,\alpha_2)=}{0,0}\\
	\isp{(\alpha_1,\alpha_2)=}{0,0}\\
\Rightarrow \text{Por igualdad de tuplas } \alpha_1 = 0 = \alpha_2\\
}
As\'{i},$B$ es linealmente independiente.


 



















































%Ahora veamos la independencia lineal para:\\ \is{B_3=}{(1,0)(0,1),(0,0)}:
\be{align*}{
	\text{Proponemos } \alpha_1,\alpha_2,\alpha_3\in \mb{R}  \text{ tal que:}\\
	\alpha_1(1,0)+\alpha_2(0,1)+\alpha_3(0,0) = (0,0)\\
	(\alpha_1,0)+(0,\alpha_2)+(0,0) = (0,0)\\
	(\alpha_1,\alpha_2) = (0,0)\\
	\Rightarrow \alpha_1=\alpha_2=0;\alpha_3\in \mb{R}\\
	\Rightarrow \text{Tiene infinidad de soluciones} \\
	\Rightarrow B_3 eslinealmente dependiente\\
	\therefore \text{No es una base}
}

Ahora veamos si $B_4$ es una base para $\mb{R}^2$:\\ \is{B_4=}{(1,0)(0,1),(6,-\frac{1}{2})}:
\be{align*}{
	\text{Proponemos } (x,y) \in \mb{R}^2 \land \\ c_1,c_2,c_3 \subseteq \mb{R} \text{ tal que:} \\
	Resolver en casa
}
\be{defin}{(Dimensi\'{o}n de un espacio vectorial)\\
	Sea $V$ un espacio vectorial sobre $\mb{K}$ y sea \is{B=}{v_1,...,v_n} una base de $V$.\\
	Entonces el n\'{u}mero de vectores de la base se denomina dimensi\'{o}n del espacio y se escrime $dimV =n$

	Obeservaci\'{o}n: Por convenci\'{o}n el espacio vectorial \is{V}{0} es el \'{u}nico espacio que carece de base(puede ser tomado como definici\'{o}n) 


}


\be{conv}{Latice o reticula.Investigar.\\ 
Ya vimos que el conjunto $B$ forma una base, 
Al realizar una transformacion, lo que se hace es deformar cosas. Si tenemos un vector en R2 nos va a devolver un R2. Este tipos de transformaciones se utilizan para mover una imagen vectorial(no en escala de bits). Sin embargo esto es complicado de operar. el producto de la transformacion es la composicion de funciones, lo cual es dificil de programar y costoso. Aplicar una regla de este estilo a mil puntos es, ineficiente, dificil e incluso peligroso computacionalmente. Para solucionar esto aprovechamos una matriz. (e isomorfismo). Sin embargo con una matriz muy grande es dificil operar los elementos, esto tambien es complicado computacionalmente. Para ayudarnos utilizaremos la diagonalizacion, es decir, el que una matriz  sea equivlente a una matriz diagonal, que representa al mismo operador en otra base. Una alternativa son los bloques diagonales, que depende del orden en que se toman los elementos de l abase. Para lograrlo podemos utiliar un reordenamiento de la base. es decir, es importante el orden de la base. es decir, {} \neq ()

}
	


















































\end{document}
%\newtheorem{teo}{Teorema}

%b - br
% \[  \
%%v - v
% \[  \
 %V - d
%
% \[  \
%p - pa
% \[  \

%\[ \le
%
%
%
%
%
%
%

