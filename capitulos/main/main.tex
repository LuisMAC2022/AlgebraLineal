\documentclass{IEEEtran}


\usepackage{enumitem}
\usepackage{hyperref}
\usepackage{amssymb}
\usepackage{amsmath}

\title{Algebra Lineal}
\author{Profesor: Jes\'{u}s Gil Galindo Cuevas\\Notas: LaRana}
\date{Enero - Febrero de 2025}
\newcommand{\is}[2]{\ensuremath{{#1}\{{#2}\}}}
\newcommand{\isp}[2]{\ensuremath{{#1}({#2})}}
\newcommand{\be}[2]{\begin{#1}{#2}\end{#1}}
\newcommand{\mb}[1]{\mathbb{#1}}
\newcommand{\ul}[1]{\underline{#1}}
\newcommand{\ol}[1]{\overline{#1}}
\newcommand{\sumin}{\sum_{i=1}^n}
\newtheorem{teo}{Teorema}
\newtheorem{defin}{Definici\'{o}n}
\newtheorem{poof}{Demostraci\'{o}n}
\newtheorem{ejem}{Ejemplo}
\newtheorem{propo}{Proposici\'{o}n}

\begin{document}
\maketitle

\documentclass{IEEEtran}


\usepackage[spanish]{babel}
\usepackage{enumitem}
\usepackage{hyperref}



\title{Algebra Lineal}
\author{Profesor: Jes\'{u}s Gil Galindo Cuevas\\Notas: Luis Daniel Arana}
\date{Febrero 2024}

%\newcommand{}{}
%\newtheorem

\begin{document}
\maketitle
\section*{Temario}
\begin{enumerate}
	\item Espacios Vectoriales
	\begin{enumerate}[label*=\arabic*]
	\item Campos	
	\item Espacios
	\item Subespacios
	\item Independencia Lineal
	\item Base
	\item Dimensi\'{o}n
\end{enumerate}
	\item Transformaciones Lineales
\begin{enumerate}[label*=\arabic*]
	\item Definici\'{o}n y propiedades
	\item Isomorfismo
	\item N\'{u}cleo e imagen
	\item Representaci\'{o}n Matricial
\end{enumerate}
	\item Eigenvalor y eigenvector
\begin{enumerate}[label*=\arabic*]
	\item Definiciones y polinomio caracteristico 
	\item Teorema de Cayley Hamilton
	\item Diagonalizaci\'{o}n de operaciones lineales
\end{enumerate}
	\item Producto Interno
\begin{enumerate}[label*=\arabic*]
	\item Espacios con producto interno 
	\item Ortogonalizaci\'{o}n  
	\item Proceso GramSchmidt
\end{enumerate}
\item Operadores Ortogonales
\begin{enumerate}[label*=\arabic*]
	\item Diagonalizaci\'{o}n
	\item Operadores sim\'{e}tricos
	\item Operadores hermitianos
\end{enumerate}
\end{enumerate}
%\input{carpeta/capitulo}
\section*{Literatura}
\begin{itemize}
	\item Larson
	\item Anton
	\item Grossman
	\item Lipschutz
	\item Friedberg
	\item Hoffman
	\item Sheldon Axler
	\item Egor Maximenko, Algebra 2 y 3
\end{itemize}
\section*{Recursos adicionales}
\begin{itemize}
\item\href{jgwalneko.github.io}{Pagina web del profesor }
\item\href{youtube.com/@jg_walneko}{Canal de Youtube }
\end{itemize}
\end{document}



\newpage

%%\documentclass{IEEEtran}


%\usepackage{amssymb}

%\usepackage{amsmath}

%\title{Introducci\'{o}n}
%\author{}
%\date{27 de Enero de 2025}

%\newcommand{\mb}[1]{\mathbb{#1}}
%\newcommand{\ul}[1]{\underline{#1}}
%\newcommand{\ol}[1]{\overline{#1}}
%\newcommand{\sumin}{\sum_{i=1}^n}
%\newtheorem{teo}{Teorema}
%\newtheorem{poof}{Demostraci\'{o}n}
%\newtheorem{teo}{Teorema}

%b - brackets matrix/
% \[  \begin{bmatrix} 0 & 0 & 0\\ 0 & 0 & 0 \end{bmatrix} \]
%%v - vertical line matrix/
% \[  \begin{vmatrix} 0 & 0 & 0\\ 0 & 0 & 0 \end{vmatrix} \]
 %V - double vertical line  matrix/
%
% \[  \begin{Vmatrix} 0 & 0 & 0\\ 0 & 0 & 0 \end{Vmatrix} \]
%p - parenthesis matrix/
% \[  \begin{pmatrix} 0 & 0 & 0\\ 0 & 0 & 0 \end{pmatrix} \]

%\[ \left\{ \begin{array}{ccc}  \end{array}        \right\} \]
%

%\begin{document}
%\maketitle
\onecolumn
\section*{Introducci\'{o}n}
Iniciaremos el curso estudiando la estructura matem\'{a}tica del  espacio vectorial.\\ Para esto utilizaremos lenguaje de teor\'{i}a de conjuntos.\\Un espacio vectorial puede ser definido como una estructura:
\begin{align*}
(V, F, +, \cdot)\\
V \text{ es un vector. }\\
F \text{ es un campo.}\\
+ \text{ es un operador.}\\
\cdot \text{ es una funci\'{o}n.}\\
+ \land \cdot \text{ son binarios.}
\end{align*}
Desarrollaremos las siguientes propiedades:

\begin{enumerate}
	\item \begin{align*}u+v \in F\end{align*}
	\item \begin{align*}u+v = v+u\end{align*}
	\item \begin{align*}v + (v + w) = (u + v) + w\end{align*}
	\item \begin{align*}\exists! 0 \in F \to u + 0 = 0 + u = u\end{align*}
	\item \begin{align*}\forall u \in F \exists! -u \in F \to  u + (-u) = -u + u = 0 \end{align*}
	\item \begin{align*}uv \in F\end{align*}
	\item \begin{align*}uv = vu\end{align*}
	\item \begin{align*}u(vw) = (uv)w\end{align*}
	\item \begin{align*}\exists! 1\in F \to u\cdot 1 = \cdot 1 = u\end{align*}
	\item \begin{align*}\forall u \in F; u \neq 0 \exists! u^{-1}\in F \to uu^{-1} = u^{-1}u = 1\end{align*}
	\item \begin{align*}u(v+w)=uv+uw \land (v+w)u = vu + wu \end{align*}
	
\end{enumerate}
%input{carpeta/capitulo}
%\end{document}


%\documentclass{IEEEtran}


%\usepackage{amssymb}

%\usepackage{amsmath}

%\title{Introducci\'{o}n}
%\author{}
%\date{27 de Enero de 2025}
%
%\newcommand{\mb}[1]{\mathbb{#1}}
%\newcommand{\ul}[1]{\underline{#1}}
%\newcommand{\ol}[1]{\overline{#1}}
%\newcommand{\sumin}{\sum_{i=1}^n}
%\newtheorem{teo}{Teorema}
%\newtheorem{poof}{Demostraci\'{o}n}
%%\newtheorem{teo}{Teorema}
%
%%b - brackets matrix/
%% \[  \begin{bmatrix} 0 & 0 & 0\\ 0 & 0 & 0 \end{bmatrix} \]
%%%v - vertical line matrix/
%% \[  \begin{vmatrix} 0 & 0 & 0\\ 0 & 0 & 0 \end{vmatrix} \]
% %V - double vertical line  matrix/
%%
%% \[  \begin{Vmatrix} 0 & 0 & 0\\ 0 & 0 & 0 \end{Vmatrix} \]
%%p - parenthesis matrix/
%% \[  \begin{pmatrix} 0 & 0 & 0\\ 0 & 0 & 0 \end{pmatrix} \]
%
%%\[ \left\{ \begin{array}{ccc}  \end{array}        \right\} \]
%%
%
%\begin{document}
%\maketitle
\section*{Introducci\'{o}n}
Iniciaremos el curso estudiando la estructura matem\'{a}tica del  espacio vectorial. Para esto utilizaremos lenguaje de teor\'{i}a de conjuntos.\\Un espacio vectorial puede ser definido como una estructura:
\begin{align*}
(V, F, +, \cdot)\\
V \text{ es un vector. }\\
F \text{ es un campo.}\\
+ \text{ es un operador.}\\
\cdot \text{ es una funci\'{o}n.}\\
+ \land \cdot \text{ son binarios.}
\end{align*}
Desarrollaremos las siguientes propiedades:

\begin{enumerate}
	\item \begin{align*}u+v \in F\end{align*}
	\item \begin{align*}u+v = v+u\end{align*}
	\item \begin{align*}v + (v + w) = (u + v) + w\end{align*}
	\item \begin{align*}\exists! 0 \in F \to u + 0 = 0 + u = u\end{align*}
	\item \begin{align*}\forall u \in F \exists! -u \in F \to  u + (-u) = -u + u = 0 \end{align*}
	\item \begin{align*}uv \in F\end{align*}
	\item \begin{align*}uv = vu\end{align*}
	\item \begin{align*}u(vw) = (uv)w\end{align*}
	\item \begin{align*}\exists! 1\in F \to u\cdot 1 = \cdot 1 = u\end{align*}
	\item \begin{align*}\forall u \in F; u \neq 0 \exists! u^{-1}\in F \to uu^{-1} = u^{-1}u = 1\end{align*}
	\item \begin{align*}u(v+w)=uv+uw \land (v+w)u = vu + wu \end{align*}
	
\end{enumerate}
%input{carpeta/capitulo}
%\end{document}



\newpage
\twocolumn
%\documentclass{IEEEtran}
%
%
%\usepackage{amssymb}
%
%\usepackage{amsmath}
%
%\title{}
%\author{Notas: Luis Daniel Arana}
%\date{Febrero de 2025}
%
%\newcommand{\mb}[1]{\mathbb{#1}}
%\newcommand{\ul}[1]{\underline{#1}}
%\newcommand{\ol}[1]{\overline{#1}}
%\newcommand{\sumin}{\sum_{i=1}^n}
%\newtheorem{teo}{Teorema}
%\newtheorem{poof}{Demostraci\'{o}n}
%%\

%b 
% \
%%v
% \
 %V
%
% \
%p 
% \


%\begin{document}
\section{Axiomas del Campo}
\begin{description}
	\item[Cerradura de la suma del campo]$\forall u,v \in V;\\ u+v \in V$
	\item[Asociatividad de la suma del campo]$\forall u,v,w \in V;\\ u+(v+w) = (u+v)+w $ 
	\item[Conmutatividad de la suma del campo]$\forall u,v \in V;\\ u+v = v+u $
	\item[Elemento neutro en el campo]$\exists! 0 \in V$ tal que: \\ $  0 + v = v+0 = v $
	\item[Inverso aditivo en el campo]$\forall v \in V \  \exists! w \in V$ tal que: \\$ v + w = w + v = 0 $ 
	\item [Cerradura del producto en el campo] $\forall \lambda \in \mb{K};\forall v \in V$ ; \\ $\lambda v \in V$
	\item[Distributividad respecto escalar] $\forall \alpha , \beta \in \mb{K};$ \\ $\forall v \in V; (\alpha + \beta )v = \alpha v + \beta v)$
	\item [Distributividad respecto elemento vector] $\forall \alpha \in \mb{K};$ \\ $  \forall u,v \in V; \alpha (u+v)= \alpha u + \alpha v$  
	\item [Asociatividad] $\forall \alpha, \beta \in \mb{K};$ \\ $\forall v \in V; (\alpha\beta)v = \alpha(\beta v) $
	\item [Identidad de $\mb{K}$ respecto al producto]$\forall u \in V; \exists! 1 \in \mb{K}$\\ tal que:  $1u= u$
\end{description}	
\begin{defin}
Sean $A,B$ conjuntos.\\ Decimos que:\\
$A \subseteq B$ \\
Si: \\
$\forall x \in A \rightarrow x \in B$
\end{defin}

%\input{carpeta/capitulo}
%\end{document}



%\onecolumn
%\documentclass{IEEEtran}
%
%
%\usepackage{amssymb}
%
%\usepackage{amsmath}
%
%\title{}
%\author{Notas: Luis Daniel Arana}
%\date{Febrero de 2025}
%
%\newcommand{\mb}[1]{\mathbb{#1}}
%\newcommand{\ul}[1]{\underline{#1}}
%\newcommand{\ol}[1]{\overline{#1}}
%\newcommand{\sumin}{\sum_{i=1}^n}
%\newtheorem{teo}{Teorema}
%\newtheorem{poof}{Demostraci\'{o}n}
%%\newtheorem{teo}{Teorema}
%
%%b - brackets matrix/
%% \[  \begin{bmatrix} 0 & 0 & 0\\ 0 & 0 & 0 \end{bmatrix} \]
%%%v - vertical line matrix/
%% \[  \begin{vmatrix} 0 & 0 & 0\\ 0 & 0 & 0 \end{vmatrix} \]
% %V - double vertical line  matrix/
%%
%% \[  \begin{Vmatrix} 0 & 0 & 0\\ 0 & 0 & 0 \end{Vmatrix} \]
%%p - parenthesis matrix/
%% \[  \begin{pmatrix} 0 & 0 & 0\\ 0 & 0 & 0 \end{pmatrix} \]
%
%%\[ \left\{ \begin{array}{ccc}  \end{array}        \right\} \]

%\begin{document}
%

\section{Espacios Vectoriales}
\begin{defin}
	(Espacio Vectorial)\\
Sea: \\
$V$ un conjunto no vac\'{i}o,\\ $\mb{F}$ un campo y \\  $+: V\times V \to V;$ \\$ \cdot : \mb{F}\times V \to V$ \\
Tal que se cumplen los siguientes axiomas:

\begin{description}
	\item[Cerradura de la suma]
\item[Asociatividad de la suma]
\item[Conmutatividad de la suma]
\item[Existencia del vector cero]
\item[Existencia del inverso aditivo]
\item[Cerradura del producto escalar sobre el vector]
\item[Distributividad del producto escalar respecto a la suma del vector]
\item[Distributividad del producto escalar respecto a la suma del campo ]
\item[Existencia del neutro multiplicativo del campo]
\item[Asociatividad del producto de escalares]
\end{description}

Entonces $V$ se llama espacio vectorial sobre el campo $\mb{F}$. A los elementos de $V$ se le llama vectores.

\begin{ejem}
$$
\end{enumerate}


\end{ejem}

\end{defin}
%%\input{carpeta/capitulo}
%\end{document}
%

%\documentclass{IEEEtran}
%
%
%\usepackage{amssymb}
%
%\usepackage{amsmath}
%
%\title{}
%\author{Notas: Luis Daniel Arana}
%\date{Febrero de 2025}
%
%\newcommand{\mb}[1]{\mathbb{#1}}
%\newcommand{\ul}[1]{\underline{#1}}
%\newcommand{\ol}[1]{\overline{#1}}
%\newcommand{\sumin}{\sum_{i=1}^n}
%\newtheorem{teo}{Teorema}
%\newtheorem{poof}{Demostraci\'{o}n}
%%\newtheorem{teo}{Teorema}
%
%%b - brackets matrix/
%% \[  \begin{bmatrix} 0 & 0 & 0\\ 0 & 0 & 0 \end{bmatrix} \]
%%%v - vertical line matrix/
%% \[  \begin{vmatrix} 0 & 0 & 0\\ 0 & 0 & 0 \end{vmatrix} \]
% %V - double vertical line  matrix/
%%
%% \[  \begin{Vmatrix} 0 & 0 & 0\\ 0 & 0 & 0 \end{Vmatrix} \]
%%p - parenthesis matrix/
%% \[  \begin{pmatrix} 0 & 0 & 0\\ 0 & 0 & 0 \end{pmatrix} \]
%
%%\[ \left\{ \begin{array}{ccc}  \end{array}        \right\} \]
%%
%%
%%
%%
%%
%%
%%
%%
%
\section{Subespacios Vectoriales}
\be{defin}{(Subespacio Vectorial)\\ Sea $V$ un espacio vectorial sobre $\mb{K}$. Sea $W\subseteq V; W \neq \emptyset$. Decimos que $W$ es subespacio vectorial de $V$ si $W$ constituye un espacio vectorial por si mismo con el campo $\mb{K}$ y las mismas operaciones de $V$.} Ejemplos: \be{itemize}{\item Todo subespacio de $\mb{R}^2 \text{ sobre } \mb{KR} $ \item $\mb{R}^2$ \item \is{}{\isp{}{0,0}\in \mb{R}^2} \item Cualquier conjunto de puntos en $\mb{R}^2$ que forme una recta que pase por el origen \is{}{x,mx \in\mb{R}^2}}

\be{propo}{(Propiedades de los subespacios vectoriales)\\Sea $V$ un espacio vectorial sobre $\mb{K}$ y sea $C \in \mb{K}$ Entonces se cumplen: \be{enumerate}{\item $0v=0$\item $\alpha0 = 0$\item Si $\alpha v=0\Rightarrow  \alpha = 0 \lor v=0$\item$(-1)v= -v$} }
\be{poof}{}
\be{teo}{(Criterio de subespacio)\\ Sean $V$ un espacio vectorial sobre $\mb{K}$ y CRITERIO DIFF from Subesp?? }
%
%
%
%
%
%
%\begin{document}
%
%%\input{carpeta/capitulo}
%\end{document}
%



\end{document}
%\newtheorem{teo}{Teorema}

%b - br
% \[  \
%%v - v
% \[  \
 %V - d
%
% \[  \
%p - pa
% \[  \

%\[ \le
%
%
%
%
%
%
%

