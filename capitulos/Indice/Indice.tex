\documentclass{IEEEtran}


\usepackage[spanish]{babel}
\usepackage{enumitem}
\usepackage{hyperref}



\title{Algebra Lineal}
\author{Profesor: Jes\'{u}s Gil Galindo Cuevas\\Notas: Luis Daniel Arana}
\date{Febrero 2024}

%\newcommand{}{}
%\newtheorem

\begin{document}
\maketitle
\section*{Temario}
\begin{enumerate}
	\item Espacios Vectoriales
	\begin{enumerate}[label*=\arabic*]
	\item Campos	
	\item Espacios
	\item Subespacios
	\item Independencia Lineal
	\item Base
	\item Dimensi\'{o}n
\end{enumerate}
	\item Transformaciones Lineales
\begin{enumerate}[label*=\arabic*]
	\item Definici\'{o}n y propiedades
	\item Isomorfismo
	\item N\'{u}cleo e imagen
	\item Representaci\'{o}n Matricial
\end{enumerate}
	\item Eigenvalor y eigenvector
\begin{enumerate}[label*=\arabic*]
	\item Definiciones y polinomio caracteristico 
	\item Teorema de Cayley Hamilton
	\item Diagonalizaci\'{o}n de operaciones lineales
\end{enumerate}
	\item Producto Interno
\begin{enumerate}[label*=\arabic*]
	\item Espacios con producto interno 
	\item Ortogonalizaci\'{o}n  
	\item Proceso GramSchmidt
\end{enumerate}
\item Operadores Ortogonales
\begin{enumerate}[label*=\arabic*]
	\item Diagonalizaci\'{o}n
	\item Operadores sim\'{e}tricos
	\item Operadores hermitianos
\end{enumerate}
\end{enumerate}
%\input{carpeta/capitulo}
\section*{Literatura}
\begin{itemize}
	\item Larson
	\item Anton
	\item Grossman
	\item Lipschutz
	\item Friedberg
	\item Hoffman
	\item Sheldon Axler
	\item Egor Maximenko, Algebra 2 y 3
\end{itemize}
\section*{Recursos adicionales}
\begin{itemize}
\item\href{jgwalneko.github.io}{Pagina web del profesor }
\item\href{youtube.com/@jg_walneko}{Canal de Youtube }
\end{itemize}
\end{document}

