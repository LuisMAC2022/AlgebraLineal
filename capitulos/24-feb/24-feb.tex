\documentclass{IEEEtran}


\usepackage{amssymb}

\usepackage{amsmath}

%\title{}
%\author{Notas: Luis Daniel Arana}
%\date{Febrero de 2025}

%\newcommand{}{}
%\newcommand{\sumin}{\sum_{i=1}^n}
\newcommand{\mb}[1]{\mathbb{#1}}
%\newtheorem{defin}{Definici\'{o}n}
\newtheorem{defin}{Definici\'{o}n} 
\newtheorem{conv}{Conversaci\'{o}n} 
\newtheorem{solu}{Soluci\'{o}n} 
\newtheorem{ejer}{Ejercicio} 

\newtheorem{ejem}{Ejemplo}

%\input{carpeta/capitulo}

\begin{document}


\begin{defin}

Sean $V$ un espacio vectorial sobre $\mb{K}, W$ un subespacio de $V$ y $S={v_1,...,v_k}$ subconjunto de $V$.

Decimos que $S$ genera a $W$ si todo $w\in W$ se escribe como combinaci\'{o}n lineal de los vectores de $S$.

\begin{ejem}

	Determine si $S={v_1, v_2}\subset \mb{R}^3$ \\
	genera a $W = {(2x+3y,x,y)\in \mb{R}^3}$ \\
	con $v_1 = (2,1,0) \land v_2=(3,0,1).$

\begin{solu}
	Sea $W=(2x+3y,x,y)\in W$
	Proponemos $c_1, c_2 \in \mb{R}$\\
	Tales que $w=c_1v_1+c_2v_2$\\
	$(2x+3y,x,y)=c_1(2,1,0)+c_2(3,0,1)$\\
	$(2x+3y,x,y)=(2c_1,1c_1,0c_1)+(3c_2,0c_2,1c_2)$\\
	$(2x+3y,x,y)=(2c_1+3c_2,c_1,c_2)$\\
	Por igualdad de tuplas tenemos $2c_1+3c_2=2x+3y;c_1 =x; c_2=y$
$\therefore S$ genera a $W$ 
\end{solu}
\end{ejem}

\begin{ejem}

	Sea $W=\{(x,y,0)\in \mb{R}^3\}$\\
	Y $S={v_1 =(1,1,1), v_2=(1,0,0)}$\\
	Determine si $S$ genera a $W$.
\begin{solu}\\
	Supongamos que $S$ genera a $W$.
	Sean $w=(x,y,0)\in W \land \lambda_1, \lambda_2 \in \mb{R}$
	Tales que $w = \lambda_1, v_1+ \lambda_2v_2 $\\
	$(x,y,0)=\lambda(1,1,1)+\lambda(1,0,0)$
	$(x,y,0)=\lambda(\lambda_1,\lambda_1,\lambda_1)+(\lambda_2,0,0)$\\
	$(x,y,0)=\lambda(\lambda_1 +\lambda_2,\lambda_1,\lambda_1)\to \left \{ \lambda_1 + \lambda_2 = x ; \lambda_1 = y; \lambda_1 = 0 \right \} \to \lambda_1 = y = 0$\\
Pero si y=0, entonces $S$ no genera a $W$. 
Contra ejemplo si $W=(1,1,0)\in W$
Supongamos que existen $a_1, a_2 \in \mb{R} $ tal que  $w=a_1v_1 + a_2v_2$\\
$(1,1,0)=a_1(1,1,1)+a_2(1,0,0)$\\
$(1,1,0)=(a_1 + a_2, a_1, a_1)$\\
\end{solu}
\end{ejem}
\end{defin}


\begin{defin} Dependencia e independencia lineal \\
	Sea $S = \{v_1,...,v_k\}\subset V$ Se dice que el conjunto de vectores $S$ es linealmente independiente si la ecuaci\'{o}n vectorial $c_1v_1+...+c_kv_k=0$
	con $c_1,...c_k \in \mb{K}$\\
	admite como soluci\'{o}n \'{u}nica:\\
	$c_1=c_2=...=c_k=0$ Este es el vector 0.
	Si hay soluciones no triviales se dice que el conjunto de vectores $S$ es linealmente dependiente.
	
\end{defin}

\begin{ejem}

\begin{enumerate}
	\item Determine si $S = \{v_1=(2,3),v_2=(1,4)\}\subset \mb{R}^2$ es un conjunto linealmente independiente.

\begin{solu}
Proponemos $c_1,c_2 \in \mb{R}$ tal que $c_1v_1 + c_2v_2 = 0$\\
$c_1(2,3)+c_2(1,4)=(0,0)$\\
$(2c_1,3c_1)+(1c_2,4c_2)=(0,0)$\\
$(2c_1 + c_2 ,3c_1+4c_2)=(0,0)$\\


Calculamos el determinante de la matriz de coeficientes asociada al sistema homog\'{e}neo:
determinante = $\left | 2 , 1  , 3, 4	\right |$

Como determinante $\neq 0$ el sistema homog\'{e}neo tiene la soluci\'{o}n \'{u}nica: 
$c_1 = c_2 = 0 $

Por tanto, $S$ es un conjunto de vectores linealmente independiente. No es necesario resolver el sistema pero se queda de tarea. 
\end{solu}

\item Determine la dependencia o indepencia lineal del conjunto de vectores $S=\{(1,1,0),(0,2,-1),(1,3,1)\} \subset \mb{R}^3$


\begin{solu}
	Sean $c_1,c_2,c_3 \in \mb{R}$ tal que\\
	$c_1v_1+c_2v_2+c_3v_3 = 0 $\\
	$c_1(1,1,0)+c_2(0,2,-1)+c_3(1,3,-1)=(0,0,0) $\\
	$(c_1+c_3,c_1+2c_2,3c_3,-c_2 -c_3)=(0,0,0) $\\

	Resolver de tarea...

	Investigar como realizar un entorno de determinantes y matrices

\end{solu}

\end{enumerate}
\end{ejem}
\begin{conv}
	para saber si un conjunto de vectores es linealmente independientes escribes al vector 0 como una combinacion lineal de ellos. esto es un sistema de ecuaciones de tipohomogeneo, por tanto tiene solucion siempre, y las soluciones pueden ser infinitas o puede ser la trivial $Ax=0$

En la independencia lineal nos interesa que la solucion sea dada solo por la solucion trivial, es decir, que no haya una infinidad

Cuando tengo un sistema de ecuaciones, cualquiera, y el determinante es cero en automatico tengo solucion unica, pero si el determinante de dicho sistema hay dos posibilidades segun la naturaleza del sistema, infinidad y trivial.

Si es homogeneo siempre es soluci\'{o}n. Si el det es 0 estoy en el caso de infinidad de soluciones. Por lo tanto el conjunto S es linealmente dependiente

Para resolver la tarea se puede utilizar la eliminaci\'{o}n Gaussiana. Transformamos la matriz a la manera escalonada
 
Transformamos la matriz (1,0,1)(1,2,3) (0,-1,1) a (1,0,1)(0,1,1)(0,0,0) Una vez escalonado despejamos para obtener las soluciones(0,0,0) Una vez escalonado despejamos para obtener las soluciones . No es necesario hacerlo, pero podemos comprobar. 

Dado esto encontramos que hay infinidad de soluciones y por tanto el sistema es dependiende.
\end{conv}
\begin{ejer}
	Sea $H=\{ f:[a,b]\to\mb{R}|f(x)|\leq 5\}$\\
	Investigue si $H$ es un espacio vectorial.
	Digamos que $A \neq 0, A \subset \mb{R}$\\
	$A$ est\'{a} acotado superiormente si existe $M \in  \mb{R} \forall a\in A$\\
tal que $a \leq M$	

\end{ejer}


\begin{conv}

si tengo un inervalo cerrado a b el infimo y el maximo coincide con la cota inferior y superior del conjunto respectivamente

si tengo un intervalo abierto a y b se acercan pero no estan definidos

me interesan los conjuntos acotados, es decir, son cerrados en el infimo y el maximo.

Una funcion acotada es tal que el conjunto imagen debe ser un conjunto acotado.

No me interesa el intervalo ab, me interesan las imagenes del intervalo.  cuando x esta en el intervalo del dominio aparece una imagen en el eje y. si esta acotado entonces la funcion esta acotada. graficamente se puede ver si yo puedo trazar una banda y todas las bandas son horizontales entonces esta acotado.
 
Un ejemplo seria el seno, el coseno, una funcion constante.
cuando un conjunto esta acotado hay una doble desigualdad, pero no necesariamente son simetricas, por lo que se suele tomar una de las dos, la mas grande, el valor absoluto para escribir la desigualdad doble como una desigualdad con valor absoluto. 

una propiedad seria que si a es menor igual que b entonces a es mayor igual que menos b 

par resolver este ejercicio usamos el valor absoluto de menos 5podemos usar la teoria de subespacios. si comprobamos que es un subespacio del conjunto...(debe satisfacer los axiomas del campo vectorial) adelanto: no lo hace 

en el eje x colocamos a,b  no importa donde
los valores de la grafica deben estar entre -5 y 5
la grafica de aquellas funciones que esten en el intervalo 
si la grafica pertenece la la funcion f entonces la funcion esta en H

aqui demostraremos que no se cumplen las cerraduras.

podemos hacerlo dando dos funciones y que la suma se salga, por ejemplo,dando dos constantes f(x)= 4, g(x) = 3.9. Ambos estan en h, si los sumamos el resultado se sale del espacio vectorial (7.9) 


\end{conv}
\end{document}

