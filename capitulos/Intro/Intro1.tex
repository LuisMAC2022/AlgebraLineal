%\documentclass{IEEEtran}


%\usepackage{amssymb}

%\usepackage{amsmath}

%\title{Introducci\'{o}n}
%\author{}
%\date{27 de Enero de 2025}

%\newcommand{\mb}[1]{\mathbb{#1}}
%\newcommand{\ul}[1]{\underline{#1}}
%\newcommand{\ol}[1]{\overline{#1}}
%\newcommand{\sumin}{\sum_{i=1}^n}
%\newtheorem{teo}{Teorema}
%\newtheorem{poof}{Demostraci\'{o}n}
%\newtheorem{teo}{Teorema}

%b - brackets matrix/
% \[  \begin{bmatrix} 0 & 0 & 0\\ 0 & 0 & 0 \end{bmatrix} \]
%%v - vertical line matrix/
% \[  \begin{vmatrix} 0 & 0 & 0\\ 0 & 0 & 0 \end{vmatrix} \]
 %V - double vertical line  matrix/
%
% \[  \begin{Vmatrix} 0 & 0 & 0\\ 0 & 0 & 0 \end{Vmatrix} \]
%p - parenthesis matrix/
% \[  \begin{pmatrix} 0 & 0 & 0\\ 0 & 0 & 0 \end{pmatrix} \]

%\[ \left\{ \begin{array}{ccc}  \end{array}        \right\} \]
%

%\begin{document}
%\maketitle
\onecolumn
\section*{Introducci\'{o}n}
Iniciaremos el curso estudiando la estructura matem\'{a}tica del  espacio vectorial.\\ Para esto utilizaremos lenguaje de teor\'{i}a de conjuntos.\\Un espacio vectorial puede ser definido como una estructura:
\begin{align*}
(V, F, +, \cdot)\\
V \text{ es un vector. }\\
F \text{ es un campo.}\\
+ \text{ es un operador.}\\
\cdot \text{ es una funci\'{o}n.}\\
+ \land \cdot \text{ son binarios.}
\end{align*}
Desarrollaremos las siguientes propiedades:

\begin{enumerate}
	\item \begin{align*}u+v \in F\end{align*}
	\item \begin{align*}u+v = v+u\end{align*}
	\item \begin{align*}v + (v + w) = (u + v) + w\end{align*}
	\item \begin{align*}\exists! 0 \in F \to u + 0 = 0 + u = u\end{align*}
	\item \begin{align*}\forall u \in F \exists! -u \in F \to  u + (-u) = -u + u = 0 \end{align*}
	\item \begin{align*}uv \in F\end{align*}
	\item \begin{align*}uv = vu\end{align*}
	\item \begin{align*}u(vw) = (uv)w\end{align*}
	\item \begin{align*}\exists! 1\in F \to u\cdot 1 = \cdot 1 = u\end{align*}
	\item \begin{align*}\forall u \in F; u \neq 0 \exists! u^{-1}\in F \to uu^{-1} = u^{-1}u = 1\end{align*}
	\item \begin{align*}u(v+w)=uv+uw \land (v+w)u = vu + wu \end{align*}
	
\end{enumerate}
%input{carpeta/capitulo}
%\end{document}

