%\documentclass{IEEEtran}
%
%
%\usepackage{amssymb}
%
%\usepackage{amsmath}
%
%\title{}
%\author{Notas: Luis Daniel Arana}
%\date{Febrero de 2025}
%
%\newcommand{\mb}[1]{\mathbb{#1}}
%\newcommand{\ul}[1]{\underline{#1}}
%\newcommand{\ol}[1]{\overline{#1}}
%\newcommand{\sumin}{\sum_{i=1}^n}
%\newtheorem{teo}{Teorema}
%\newtheorem{poof}{Demostraci\'{o}n}
%%\

%b 
% \
%%v
% \
 %V
%
% \
%p 
% \


%\begin{document}
\section{Axiomas del Campo}
\begin{description}
	\item[Cerradura de la suma]$\forall u,v \in V;\\ u+v \in V$
	\item[Asociatividad de la suma]$\forall u,v,w \in V;\\ u+(v+w) = (u+v)+w $ 
	\item[Conmutatividad de la suma]$\forall u,v \in V;\\ u+v = v+u $
	\item[Elemento neutro]$\exists! 0 \in V$ tal que: \\ $  0 + v = v+0 = v $
	\item[Inverso aditivo]$\forall v \in V \  \exists! w \in V$ tal que: \\$ v + w = w + v = 0 $ 
	\item [Cerradura del producto] $\forall \lambda \in \mb{K};\forall v \in V$ ; \\ $\lambda v \in V$
	\item[Distributividad respecto escalar] $\forall \alpha , \beta \in \mb{K};$ \\ $\forall v \in V; (\alpha + \beta )v = \alpha v + \beta v)$
	\item [Distributividad respecto elemento vector] $\forall \alpha \in \mb{K};$ \\ $  \forall u,v \in V; \alpha (u+v)= \alpha u + \alpha v$  
	\item [Asociatividad] $\forall \alpha, \beta \in \mb{K};$ \\ $\forall v \in V; (\alpha\beta)v = \alpha(\beta v) $
	\item [Identidad de $\mb{K}$ respecto al producto]$\forall u \in V; \exists! 1 \in \mb{K}$\\ tal que:  $1u= u$
\end{description}	
\begin{defin}
Sean $A,B$ conjuntos.\\ Decimos que:\\
$A \subseteq B$ \\
Si: \\
$\forall x \in A \rightarrow x \in B$
\end{defin}

%\input{carpeta/capitulo}
%\end{document}

