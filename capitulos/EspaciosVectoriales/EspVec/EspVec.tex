%\documentclass{IEEEtran}
%
%
%\usepackage{amssymb}
%
%\usepackage{amsmath}
%
%\title{}
%\author{Notas: Luis Daniel Arana}
%\date{Febrero de 2025}
%
%\newcommand{\mb}[1]{\mathbb{#1}}
%\newcommand{\ul}[1]{\underline{#1}}
%\newcommand{\ol}[1]{\overline{#1}}
%\newcommand{\sumin}{\sum_{i=1}^n}
%\newtheorem{teo}{Teorema}
%\newtheorem{poof}{Demostraci\'{o}n}
%%\newtheorem{teo}{Teorema}
%
%%b - brackets matrix/
%% \[  \begin{bmatrix} 0 & 0 & 0\\ 0 & 0 & 0 \end{bmatrix} \]
%%%v - vertical line matrix/
%% \[  \begin{vmatrix} 0 & 0 & 0\\ 0 & 0 & 0 \end{vmatrix} \]
% %V - double vertical line  matrix/
%%
%% \[  \begin{Vmatrix} 0 & 0 & 0\\ 0 & 0 & 0 \end{Vmatrix} \]
%%p - parenthesis matrix/
%% \[  \begin{pmatrix} 0 & 0 & 0\\ 0 & 0 & 0 \end{pmatrix} \]
%
%%\[ \left\{ \begin{array}{ccc}  \end{array}        \right\} \]

%\begin{document}
%

\section{Espacios Vectoriales}
\begin{defin}
	(Espacio Vectorial)\\
Sea: \\
$V$ un conjunto no vac\'{i}o,\\ $\mb{F}$ un campo y \\  $+: V\times V \to V;$ \\$ \cdot : \mb{F}\times V \to V$ \\
Tal que se cumplen los siguientes axiomas:

	$\forall u,v,w \in V; \alpha, \beta \in \mb{F}$
\begin{description}
	\item[Cerradura de la suma en el espacio vectorial.]\ \\$u+v \in V $
	\item[Asociatividad de la suma en el espacio vectorial.]\ \\\isp{}{u+w}\isp{= u +}{v+w}
\item[Conmutatividad de la suma en E.V.]\ \\$u+v = v+u$
\item[Existencia del vector cero.]$\exists! 0 \in V$ tal que:\\
	$0+v=v+0=v$
\item[Existencia del inverso aditivo del vector.]$\exists! -v\in V$ tal que:\\
	$-v+v=v+(-v)=0$
\item[Cerradura del producto escalar sobre el vector.]\ \\$\alpha v \in V$
\item[Distributividad del P.E. respecto a la suma del vector.]\ \\\isp{\alpha}{u+v}$=\alpha u + \alpha v $
\item[Distributividad del P.E. respecto a la suma del campo.]\ \\\isp{}{\alpha + \beta}$v = \alpha v + \beta v$
\item[Existencia del neutro multiplicativo del campo.]$\exists!1\in\mb{F}$ tal que: \\
	$1\cdot v = v$
\item[Asociatividad del producto de escalares por el vector.]\ \\ \isp{\alpha}{\beta v}\isp{=}{\alpha\beta}$v$
\end{description}

Entonces $V$ se llama espacio vectorial sobre el campo $\mb{F}$. A los elementos de $V$ se le llama vectores.

\end{defin}

\be{ejem}{$\mb{R}^n$ sobre $\mb{R}$ es un espacio vectorial con las operaciones:
	\be{align*}{+: \mb{R}^n \times \mb{R}^n \to \mb{R}^n\\\cdot: \mb{R} \times \mb{R}^n \to \mb{R}^n} 
	Sean \isp{x}{x_1,x_2,...,x_n}, \isp{y}{y_1,...,y_n}$\in \mb{R}^n$\\
	\isp{x+y:}{x_1+y_1,x_2+y_2,...,x_n+y_n}$\in\mb{R}^n$\\
	\isp{\alpha x :}{\alpha x, \alpha x_1,...,\alpha x_n  }$\in\mb{R}^n$\\
	\be{enumerate}{
	\item Cerradura de suma.\\
		Comprueba: \isp{x+y=}{x_1+y_1,...,x_n+y_n}$\in \mb{R}^n$
		\be{align*}{t_1=x_1 + y_1, t_2 = x_2 + y_2, ..., t_n = x_n + y_n \\ \Rightarrow \isp{x+y=}{ t_1,t_2,...,t_n}\mb{R}^n\\ t_k \in \mb{R} \text{ por cerradura de la suma en el campo}\\ \text{Siendo } k=\sumin} 
	\item Conmutatividad de suma.
	\item Asociatividad de suma.
	\item Existencia del vector 0 en $\mb{R}^n$.
	\item Existencia del inverso aditivo.
	\item Cerradura del producto por escalar.
	\item Distributividad del P.E respecto a suma del vector.
	\item Distributividad del P.E respecto a suma del campo.
	\item Existencia del neutro multiplicativo del campo.
\item Asociatividad del producto de escalares.}}

%%\input{carpeta/capitulo}
%\end{document}
%
