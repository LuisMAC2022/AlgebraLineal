%\documentclass{IEEEtran}
%
%
%\usepackage{amssymb}
%
%\usepackage{amsmath}
%
%\title{}
%\author{Notas: Luis Daniel Arana}
%\date{Febrero de 2025}
%
%\newcommand{\mb}[1]{\mathbb{#1}}
%\newcommand{\ul}[1]{\underline{#1}}
%\newcommand{\ol}[1]{\overline{#1}}
%\newcommand{\sumin}{\sum_{i=1}^n}
%\newtheorem{teo}{Teorema}
%\newtheorem{poof}{Demostraci\'{o}n}
%%\newtheorem{teo}{Teorema}
%
%%b - brackets matrix/
%% \[  \begin{bmatrix} 0 & 0 & 0\\ 0 & 0 & 0 \end{bmatrix} \]
%%%v - vertical line matrix/
%% \[  \begin{vmatrix} 0 & 0 & 0\\ 0 & 0 & 0 \end{vmatrix} \]
% %V - double vertical line  matrix/
%%
%% \[  \begin{Vmatrix} 0 & 0 & 0\\ 0 & 0 & 0 \end{Vmatrix} \]
%%p - parenthesis matrix/
%% \[  \begin{pmatrix} 0 & 0 & 0\\ 0 & 0 & 0 \end{pmatrix} \]
%
%%\[ \left\{ \begin{array}{ccc}  \end{array}        \right\} \]

%\begin{document}
%

\section{Espacios Vectoriales}
\begin{defin}
	(Espacio Vectorial)\\
Sea: \\
$V$ un conjunto no vac\'{i}o,\\ $\mb{F}$ un campo y \\  $+: V\times V \to V;$ \\$ \cdot : \mb{F}\times V \to V$ \\
Tal que se cumplen los siguientes axiomas:

\begin{description}
	\item[Cerradura de la suma]
\item[Asociatividad de la suma]
\item[Conmutatividad de la suma]
\item[Existencia del vector cero]
\item[Existencia del inverso aditivo]
\item[Cerradura del producto escalar sobre el vector]
\item[Distributividad del producto escalar respecto a la suma del vector]
\item[Distributividad del producto escalar respecto a la suma del campo ]
\item[Existencia del neutro multiplicativo del campo]
\item[Asociatividad del producto de escalares]
\end{description}

Entonces $V$ se llama espacio vectorial sobre el campo $\mb{F}$. A los elementos de $V$ se le llama vectores.

\begin{ejem}
$$
\end{enumerate}


\end{ejem}

\end{defin}
%%\input{carpeta/capitulo}
%\end{document}
%
