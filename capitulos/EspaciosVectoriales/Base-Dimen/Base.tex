\section{Base}

\be{defin}{(Base de un espacio vectorial) \\
	Sea $V$ un espacio vectorial sobre el campo $\mb{K} \text{y sea} \is{B=}{v_1,v_2,...,v_n}subset V$. 
	Se dice que $B$ es una base para $V$ si satisface:
	\be{enumerate}{
		\item$B$ genera a $V$
		\item$B$ es conjunto linealmente independiente }
		}
	\be{ejem}{$V$ es $\mb{R}^2$ En $\mb{R}^2$, sea \is{B=}{(1,0),(0,1)}\\
		Probemos que $B$ es una base para $V$.
		\\ Veamos si $B$ genera a $V$:
		\be{align*}{\text{Sea } \isp{v=}{x,y}\in\mb{R}^2. \\
			\text{Proponemos: } c_1,c_2\in\mb{R} \text{ tal que: } \\
			v=c_1v_1 + c_2v_2 \\
			(x,y)= c_1(1,0)+c_2(0,1)\\
			(x,y)= (c_1,0)+(0,c_2)\\
			(x,y)= (c_1,c_2)\\
			\text{Por igualdad de tuplas, tenemos:}\\
			}
		}
Ahora veamos la independencia lineal: \\
Proponemos $\alpha_1\alpha_2 \in \mb{R}$ tal que:
\be{align*}{
	\alpha_1v_1+\alpha_2v_2=0\\
	\isp{\alpha_1(1,0)+\alpha_2(0,1)=}{0,0}\\
	\isp{(\alpha_1,0)+(0,\alpha_2)=}{0,0}\\
	\isp{(\alpha_1,\alpha_2)=}{0,0}\\
\Rightarrow \text{Por igualdad de tuplas } \alpha_1 = 0 = \alpha_2\\
}
As\'{i},$B$ es linealmente independiente.


 

















































