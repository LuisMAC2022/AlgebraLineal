Ahora veamos la independencia lineal para:\\ \is{B_3=}{(1,0)(0,1),(0,0)}:
\be{align*}{
	\text{Proponemos } \alpha_1,\alpha_2,\alpha_3\in \mb{R}  \text{ tal que:}\\
	\alpha_1(1,0)+\alpha_2(0,1)+\alpha_3(0,0) = (0,0)\\
	(\alpha_1,0)+(0,\alpha_2)+(0,0) = (0,0)\\
	(\alpha_1,\alpha_2) = (0,0)\\
	\Rightarrow \alpha_1=\alpha_2=0;\alpha_3\in \mb{R}\\
	\Rightarrow \text{Tiene infinidad de soluciones} \\
	\Rightarrow B_3 eslinealmente dependiente\\
	\therefore \text{No es una base}
}

Ahora veamos si $B_4$ es una base para $\mb{R}^2$:\\ \is{B_4=}{(1,0)(0,1),(6,-\frac{1}{2})}:
\be{align*}{
	\text{Proponemos } (x,y) \in \mb{R}^2 \land \\ c_1,c_2,c_3 \subseteq \mb{R} \text{ tal que:} \\
	Resolver en casa
}
\be{defin}{(Dimensi\'{o}n de un espacio vectorial)\\
	Sea $V$ un espacio vectorial sobre $\mb{K}$ y sea \is{B=}{v_1,...,v_n} una base de $V$.\\
	Entonces el n\'{u}mero de vectores de la base se denomina dimensi\'{o}n del espacio y se escrime $dimV =n$

	Obeservaci\'{o}n: Por convenci\'{o}n el espacio vectorial \is{V}{0} es el \'{u}nico espacio que carece de base(puede ser tomado como definici\'{o}n) 


}


\be{conv}{Latice o reticula.Investigar.\\ 
Ya vimos que el conjunto $B$ forma una base, 
Al realizar una transformacion, lo que se hace es deformar cosas. Si tenemos un vector en R2 nos va a devolver un R2. Este tipos de transformaciones se utilizan para mover una imagen vectorial(no en escala de bits). Sin embargo esto es complicado de operar. el producto de la transformacion es la composicion de funciones, lo cual es dificil de programar y costoso. Aplicar una regla de este estilo a mil puntos es, ineficiente, dificil e incluso peligroso computacionalmente. Para solucionar esto aprovechamos una matriz. (e isomorfismo). Sin embargo con una matriz muy grande es dificil operar los elementos, esto tambien es complicado computacionalmente. Para ayudarnos utilizaremos la diagonalizacion, es decir, el que una matriz  sea equivlente a una matriz diagonal, que representa al mismo operador en otra base. Una alternativa son los bloques diagonales, que depende del orden en que se toman los elementos de l abase. Para lograrlo podemos utiliar un reordenamiento de la base. es decir, es importante el orden de la base. es decir, {} \neq ()

}
	
















































