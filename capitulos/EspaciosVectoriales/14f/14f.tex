\newpage
\begin{defin} (Combinaci\'{o}n Lineal)

    Sea:\\ $ S =\{v_1,.., v_k\} %un subconjunto de un espacio vectorial%
    \subset V  \land c_1,..,c_k \in \mathbb{K}$\\ Se dice que un vector $v \in V  $ es una combinaci\'{o}n lineal de los vectores en S si v se puede escribir como  $v =\sum{_{k=1}^nc_kv_k }$
\end{defin}



\begin{ejem}
    Sea:\\ $S=\{v_1,v_2\} \subset \mathbb{R}^2$\\ \text{donde:} \\$v_1 =(4,-3) \land v_2 = (-1,9).   $

    Escriba, de ser posible a $v=(6,-18)\in \mathbb{R}^2$ como una cominaci\'{o}n lineal de $v_1$ y $v_2$.

    Soluci\'{o}n:
    Supongamos que existen $c_1,c_2 \in \mathbb{R}$ tal que $v = c_1v_1 + c_2v_2$


    $ (6,-18) = c_1(4,-3)+c_2(-1,9) $ \\$ (6,-18) = (4c_1-c_2,-3c_1+9c_2) $ \\ Por igualdad de parejas ordenadas tenemos que: \\

    \{ $(4c_1 + 9c_2 = 6)9; -3c_1 + 9c_2=-18$\} $\to 36c_1-9c_2 = 54 ; -3c_1 + 9c_2 = -18 \to 33c_1 =36 \to c_1 = \frac{12}{11}  $ Tarea sustituir $c_1$
\end{ejem}

    \begin{ejem}

     Escriba, de ser posible, a $v=(3,\frac{1}{2})\in \mathbb{R}^2$ como una combinacion lineal de $v_1=(-2,3),v_2=(4,-6) \in \mathbb{R}^2$

    Solucion:

    Supongamos que existen $k_1,k_2 \in \mathbb{R}$ tal que $v= k_1,v_1 + k_2v_2$
    \\
    $(3,\frac{1}{2}=k_1(-2,3)+k_2(4,-6))$ \\
     $(3,\frac{1}{2}=(-2k_1,3k_1)+(4k_2,-6k_2))$ \\
      $(3,\frac{1}{2}=(-2k_1 + 4k_2,3k_1 -6k_2))$ \\

      Por igualdad de parejas ordenadas tenemos:

      $\{(-2k_1 + 4k_2=3)3; (3k_1 -6k_2 = \frac{1}{2})2\} \to \{-6k_1+12k_2=9;6k_1-12k_2=2\}\to 0=10 $

      Se llega a una contradiccion
\end{ejem}

\begin{ejem}

 Encuentre el valor de $p\in \mathbb{R} \text{ para que } v=(1,p)\in \mathbb{R}^2 $ sea combinaci\'{o}n lineal de $v_1=(2,3); v_2=(-1,4)$.

Soluci\'{o}n: Sean $\alpha,\beta \in \mathbb{R} $ tal que $v=\alpha v_1 + \beta v_2$\\ $(1,p) = \alpha(2,3)+\beta (-1,4)$\\ $(1,p) = (2\alpha,3\alpha)+ (-1\beta,4\beta)$\\

Se genera el sistema:

$\{(2\alpha-\beta = 1)4;3\alpha+4\beta = p\} \to (8\alpha - 4\beta = 4) + (3\alpha+4\beta = p) = 11\alpha = p + 4 \to \alpha = \frac{p+4}{11} $

Sustituimos

$(3\frac{p+4}{11}+4\beta = p)\to 4\beta = p -\frac{3p+12}{11} \to 4\beta = \frac{11p - 3p-12}{11} \to \beta = \frac{8p -12}{44} \to \beta = \frac{2p -3}{11}; p\in \mathbb{R} $

Si $p=7 \to \alpha = 1; \beta = 1$

Si $p=0, \text{ tenemos } \alpha=\frac{4}{11}; \beta=\frac{-3}{11}$

Soluci\'{o}n: Si $p\in \mathbb{R}, (1,p) = \frac{p+4}{11} (2,3) + \frac{2p-3}{11}(-1,4)$. Tarea: Comprueba la proposici\'{o}n anterior.
\end{ejem}



\begin{ejem}

 Escriba, de ser posible

$v = [\begin{matrix}
    1 & 2 & 3 \\ 4 & 5 & 6
\end{matrix} ] \in M_{2x2}(\mathbb{R}) $

Como una combinaci\'{o}n lineal de $v_1,v_2,v_3,v_4,\in M_{2x2}(\mathbb{R})$ donde $v_1 = [\begin{matrix}
    1 & 0  \\ 0 & -1
\end{matrix} ] , v_2 = [\begin{matrix}
    0 & 1  \\ 0 & 0
\end{matrix} ], v_3 = [\begin{matrix}
    1 & 1  \\ 1 & 0
\end{matrix} ] \text{ y } v_4 = [\begin{matrix}
    -1 & 0  \\ 0 & -1
\end{matrix} ] $


Soluci\'{o}n: Supongamos que existen $a_1,a_2,a_3,a_4\in\mathbb{R}$ tal que $v =a_1v_1+a_2v_2+a_3v_3+a_4v_4$


$    [\begin{matrix}
    2 & -3  \\  0& 4
\end{matrix} ]  = a_1[\begin{matrix}
    1 & 0  \\ 0 & -1
\end{matrix} ] + 2_2 [\begin{matrix}
    0 & 1  \\ 0 & 0
\end{matrix} ]+ a_3[\begin{matrix}
    1 & 1  \\ 1 & 0
\end{matrix} ] +  a_4[\begin{matrix}
    -1 & 0  \\ 0 & -1
\end{matrix} ] $


$\{a_1 + a_3 -a_4 = 2...(1);a_2+a_3 =-3...(2); a_3 = 0...(3); -a_1 -a_4 = 4...(4)  \} $

De (3), $a=0$. Sustituir $a_3 = 0 $ en (2): $a_2=-3$

Sustituir $a_3 = 0$ en (1) y resolver junto (4):

$(a_1 -a_4=2)+ (-a_1 -a_4=4)=-2a_4=6 \to a_4=-3; a_1=-1.  $ Tarea sustituir y comprobar.

\end{ejem}

\begin{ejem}

De ser posible, escriba a $p(x)=2x^2-6x+9$  como una combinaci\'{o}n lineal de los vectores $u(x)= 1-x, v(x)=3x^2, w(x)=5+x^2\in P_2(\mathbb{R})$

Soluci\'{o}n: Sean  $a,b,c\in\mathbb{R}$ tal que $p(x)=au(x)+bv(x)+cw(x)$

$2x^2-6x+9=a(1-x)+b(3x^2)+c(5+x^2) = a-ax+3bx^2=5c=cx^2$

$(a+5c)-ax+(3b+c)x^2$

Por igualdad de polinomios tenemos $\{2=3b+c; -6=-a \to a=6; 9 = a+5c\to c=\frac{3}{5}\}\to 3b+c=2 \to b=\frac{7}{15}$

As\'{i}: $2x^2-6x+9=6(1-x)+\frac{7}{15}(3x^2)+\frac{3}{5}(5+x^2)$

$=6-6x+\frac{21}{15}x^2+\frac{3}{5}x^2$

$=2x^2-6x+9$

\end{ejem}

$==================================================================$
\begin{teo}

Envoltura o envolvente lineal o espacio generado)

Sean $v1,..,v_k$  elementos de un espacio vectorial $V$ sobre el campo $\mathbb{K}$ . El conjunto de todas las combinaciones lineales que se pueden formar con $v_j \leq j \leq k$ , es un subespacio de $V$ y se denomina: envoltura lineal, envolvente lineal o espacio generado por los $v_j$.
\end{teo}


\begin{poof}

    Denotemos por $gen=\{v_1,..,v_k\} $ al conjunto de todas las combinaciones lineales de los $v_j$.  Demostrar que no es vacio y es subconjunto de $V$.

   Idea: Si $\alpha v_1+...+\alpha_kv_k\in V$ por cerradura; Si $\beta v_1+...+\beta_kv_k\in V$ por cerradura. Entonces $(\alpha v_1+...+\alpha_kv_k) \cdot (\beta v_1+...+\beta_k v_k)$ por cerradura del producto.

La unica combinaci\'{o}n lineal que puede formar un subespacio es el cero, que formaria el subespacio trivial.

Regresando a la demostraci\'{o}n:

    Como $v_1,..,v_k \in gen\{..\} $ entonces $gen\{..\}\neq \emptyset $ y $gen\{..\} \subset V$

    Aplicando el cr\'{i}terio de subespacio para verificar la cerradura de las operaciones:

    sean $\alpha v_1+...+\alpha_kv_k;\beta v_1+...+\beta_kv_k \in gen$ y $\lambda\in\mathbb{K}.$ Entonces:

    $(\alpha_1 v_1+...+\alpha_kv_k) + (\beta_1 v_1+...+\beta_k v_k) = (\alpha_1 + \beta_1)v_1+..+(\alpha_k+\beta_k)v_k \in gen$ por asociatividad de la suma en $V$ y distributividad de la suma en $V$   y luego por la cerradura de la suma en $\mathbb{K}$.

    La otra cerradura se deja como ejercicio.

    Ejemplo: Sea $v=(1,1)\in\mathbb{R}^2$.

    Este vector por si solo no forma un espacio vectorial. Salvo si fuera el vector 0. Sin embargo, al formar todas las combinaciones lineales, si genera un subespacio. Entonces
    $gen\{v\}$ es un subespacio de $\mathbb{R}^2$.

    Geometricamente seria todos los puntos de la diagonal que pasa por el origen(al ser espacio vectorial) y por el punto $v$. $gen\{(1,1)\}=\{(x,x)\in\mathbb{R}^2\}$.

    Otra forma de escribir el generado es: $gen\{(1,1)\}=\{\lambda(1,1):\lambda\in\mathbb{R}\}$.

    Notaci\'{o}n: Sea $S\subset V; genS =linS = \mathcal{L}(S) = span(S) = \ell(S)$

\end{poof}

Ejercicio: La union de 2 subespacios es siempre un subespacio?

$V=\mathbb{R}^2; W_1=\{(x,0)\in\mathbb{R}^2\}; W_2 =\{(0,y)\in\mathbb{R}^2\}$

Recordemos que $W_1 \cup W_2= \{ \mathbb{R};\mathbb{R}^2;\{(0,0)\}\}$

Podemos definir como $W_1 \cup W_2= (x,0) \lor (0,x) : x\in\mathbb{R}\}$

$(1,0)+(0,1) = (1,1)  \in$

La intereseccion finita de un subespacio con otro subespacio es un subespacio. Una interseccion de un infiniro numerable es un subespacio. La interseccion de un infinito no numerable de subespacios tambien es un subespacio(?). Proposici\'{o}n: La interseccion arbitraria de subespacios es un subespacio.

\begin{nota}

Estudiar infinitos numerables y Analisis matematicos de Rudin, demostracion de George Cantor sobre inumerabilidad de los numeros Reales, no se puede hacer una funcion buyectiva de los naturales.
\end{nota}

%\begin{figure}
%    \centering
%    \includegraphics[width=1\linewidth]{image.png}
%    \caption{Meme}
%    \label{fig:enter-label}
%\end{figure}


\begin{propo}



Pensemos en un espacio vectorial $V$ y un subespacio $W_1,W_2 \in V$

$W_1 \cap W_2 = \{w\in W_1  \land w \in W_2\}\subset V$
\end{propo}

\begin{propo}
La intersecci\'{o}n  arbitraria de subespacios es un subespacio.

Sea $V$ un espacio vectorial sobre $\mathbb{K}$

y sea $\{W_j\}$ una colleci\'{o}n de subespacios de $V$.

Entonces:

$\cap_{j\in J} W_j$ es un subespacio de $V$. $J$ es un conjunto arbitrario.

\end{propo}

\begin{poof}

    Como el neutro aditivo\\ (o vector cero) de $V,0_v\in W_j \forall j\in J$\\
    Entonces:\\ $0_v \in \cap_{j\in J} W_j$,\\
    luego:\\ $\cap_{j\in J} W_j \neq \emptyset$.\\
    Adem\'{a}s, todos los\\
    $W_j \subset V \to \cap W_j \subset V \forall j\in J$\\
    \textbf{Apliquemos el criterio de subespacio}\\
    Sean:
    $u,v \in \cap_{j\in J} W_j; \lambda\in\mathbb{K} $\\
    Como: $u,v \in \cap_{j\in J} W_j \to u,v \in W_j \forall j\in J$\\
    Pero $\forall j\in J,W_j $\\
    es cerrado bajo suma y producto por escalar de $V$.\\
    Entonces:\\
    $u+v\in W_j \forall j\in J \to u+v \in \cap _{j\in J} W_j$\\
    Por otro lado:\\
    $\lambda u \in W_j \forall j\in J$\\
    Luego: \\$\lambda u \in \cap_{j\in J}W_j$

\end{poof}

\begin{propo}
    El conjunto soluci\'{o}n de un sistema homog\'{e}neo es un subespacio.\\

    Sea $A\in M_{m\times n}(\mathbb{K}).$
    El conjunto soluci\'{o}n del sistema:
    $AX=0$\\
    con $X=[x_1,..,x_n]^T\in M_{n\times 1}(\mathbb{K})$
    y $0=[0,..,0]^T\in M_{m\times 1}(\mathbb{K}),$\\ es un subespacio de $\mathbb{K}^2$
\end{propo}

\begin{ejem}
   \[ \left\{ \begin{array}{c}
        x-2y+3z=0 \\
         4z=0
   \end{array}  \right\}\] Entonces: \\
    $ z=0 \land x-2y = 0; x=2y$\\
    Conjunto soluci\'{o}n:
    $S=\{(2y,y,0)\in \mathbb{R}^3\}$


\end{ejem}



